\chapter{Abstract}
This document explains how to install, configure and use the ESP Basic software package and tools.

The purpose of the ESP\_Basic is to provide a tool that allows to perform automatic tests of the on-board simulator through a set of scenarios. During development and tests phases, it is required to validate the changes and perform non regression tests.

The Test Environment is a graphical application integrating an on-board, as well as a simplified DMI. It can simulate all interfaces to the EVC (e.g. balises, radio, loops, odometer, TIU and driver interfaces). It is possible to test the internal state of on-board in order to check its correct behaviour according to an input scenario.

The EVC is included in an independent application EVC Server which can run on the same computer than the Test Environment or an other computer using TCP/IP connection.

This document only concerns ESP Basic installation, configuration and use. For Test Environment dedicated scenarios, report to Test\_Environment\_User\_guide documentation.

\emph{Warning}: At the moment, this documentation is only valid for CentOS 6.5, CentOS 6.6 (Atom computers or virtual machines), as well as Red Hat 6.4.

\chapter{Installation}
\section{Prerequisites}
\subsection{Local configuration}
In order to install ESP\_Basic in local configuration, the following are required:
\begin{itemize}
\item Computer
\item CentOS 6.5 or 6.6
\item Internet connection
\item Root access
\end{itemize}
\subsection{Remote configuration}
In order to install ESP\_Basic in remote configuration, the following are required:
\begin{itemize}
\item 2x Computers
\item 2x CentOS 6.5 or 6.6
\item Internet connection for each computer
\item Root access on each computer
\item \emph{Account name must be the same on both computers.}
\end{itemize}

\section{Installation}
\emph{NB: You will have to repeat this operation for each computer in remote configuration.}
\newline
\emph{Warning: The installation process requires a computer reboot. Save and close all your open documents before starting this step.}
\begin{itemize}
\item Copy ESP\_Basic archive on desktop
\item Open a terminal on desktop
\item Unpack ESP\_Basic archive
	\newline
	\verb|>tar xvfj openETCS_WP5_X_X_X.tar.bz2 |
\item Connect as root
\item Launch installation script
	\newline
	\verb|>./Install_Basic_ESP |
\item Wait the computer to complete its reboot
\end{itemize}

\section{Getting a license (First installation only)}
\subsection{Getting computer fingerprint}
\begin{itemize}
	\item open a \emph{non-root} terminal
	\item move to ESP\_Basic folder
	\newline
	\verb|>cd /usr/local/ESP_Basic |          
	\item move to Licensing folder
	\newline
	\verb|>cd Licensing |
	\item launch script for getting fingerprint
	\newline
	\verb|> ./echoid >> echoid_output.txt|            
	\item send an email with file echoid\_output.txt to \emph{licenseactivation@ersa-france.com}
\end{itemize}

\subsection{Installing license}

You will receive an email with a licensing file

\begin{itemize}
	\item open licensing file with a text editor
	\item open a root terminal
	\item move to ESP\_Basic folder
	\newline
	\verb|>cd /usr/local/ESP_Basic |          
	\item move to Licensing folder
	\newline	
	\verb|>cd Licensing |	
	\item launch license installer script	
	\newline	
	\verb|>./licenseinstaller -i -l 'xxxx' /opt/ERSA/license/license.rc |         where xxxx must be replaced by the key in licensing file (you can find it from first character to \# character \emph{include})	
\end{itemize}

\section{Validation}

Check that the folder \emph{/usr/local/ESP\_Basic/} exists.
\newline
Check that the following tree exists in \emph{/usr/local/ESP\_Basic/}:
\begin{itemize}
\item folder \emph{evc\_server\_gui}
\begin{itemize}
\item executable \emph{evc\_server\_gui}
\item script \emph{RunESG.py}
\item optional folder \emph{data}
\item optional compilation files
\end{itemize}
\item folder \emph{lib}
\begin{itemize}
\item shared library \emph{libevc.so}
\item some symbolic links to this library
\end{itemize}
\item folder \emph{test\_runner2}
\begin{itemize}
\item executable \emph{test\_runner2}
\item script \emph{RunTestRunner.py}
\item folder \emph{scenarios} which contains scenarios for Test Environment
\item optional compilation files
\end{itemize}
\end{itemize}

\section{Upgrade ESP\_Basic}
\begin{itemize}
\item Uninstall previous ESP\_Basic version
	\newline
	\verb|>rpm -e ESP_Basic |
\item Install new ESP\_Basic version
	\newline
	\verb|>yum localinstall ESP_Basic_X_X_X_X.rpm |
\end{itemize}

\section{Uninstall ESP\_Basic}
\begin{itemize}
\item Uninstall ESP\_Basic version
	\newline
	\verb|>rpm -e ESP_Basic |
\end{itemize}

\chapter{Automatic Configuration}
This chapter describes how to use configuration script in order to automatically configure ESP\_Basic and avoid manual configuration.
\section{Description}
This script allows to:
\begin{itemize}
\item change host name on local and remote computers if needed
\item change iptables to allow communication between EVC and test runner
\item configure ssh communication in order to avoid to enter password for each ssh communication
\end{itemize}

\section{Launching}
Script to launch Test runner can detect if a new configuration is needed (i.e local/remote hostname or ip change).
Use following command to launch configuration script
\begin{itemize} 
\item Local configuration
\newline
\verb|>python configurationScript.py -l |
\item Remote configuration
\newline
\verb|>python configurationScript.py --remote-ip 192.168.55.234 |
\end{itemize}

\section{Parameters}
\begin{itemize}
\item -h or \verb!--!help: show help message and exit
\item -d or \verb!--!debug-trace: activate debug trace for script
\item -s or \verb!--!server-side: Script launch on server (machine where ESG is running)
\item -l or \verb!--!local: Run local configuration for ESP (no need to specify remote-ip parameter)
\item \verb!--!local-password: password to connect to local host. Default value = dbm
\item \verb!--!local-hostname: hostname for local host (only use if current hostname is localhost.localdomain). Default value = ESP\_Basic2
\item \verb!--!local-root-password: password to connect as a root on local machine. Default value = password
\item \verb!--!remote-login: login to connect to remote host. Default value = dbm
\item \verb!--!remote-password: password to connect to remote host. Default value = dbm
\item \verb!--!remote-hostname: hostname for remote host (only use if current hostname is localhost.localdomain). Default value = ESP\_Basic1
\item \verb!--!remote-ip: IP address for remote host
\item \verb!--!remote-root-password: password to connect as a root on remote machine. Default value = password
\item \verb!--!remote-script-path: configuration script path on remote machine. Default value = /usr/local/ESP\_Basic/test\_runner2
\end{itemize}

\section{Warnings}
\subsection{Backup current configuration}
If user install ESP\_Basic on a non-empty computer (or virtual machine), it is highly recommended to save on a safe place the following files and folders:
\begin{itemize}
\item \emph{/etc/hosts} 
\item \emph{/etc/sysconfig/network}
\item \emph{/home/\emph{user\_name}/.ssh}
\end{itemize}
Use root access to save/restore backup file

\subsection{ESP\_Basic failure configuration}
If there is a failure during execution of configuration script, try manual configuration describe in next chapter.

\subsection{Operating System}
This configuration script is only valid for CentOS 6.5, CentOS 6.6. It should work on Red Hat 6.4 too.

\chapter{Manual Configuration}
\section{Local configuration}
This step is mandatory to be performed once only, after the installation step.
To achieve this step you have to retrieve the computer IP address (use command \emph{ifconfig} to get it) and host name (use command \emph{hostname} to get it).
\newline
For example we assume:
\begin{itemize}
\item ip address = 192.168.55.234
\item host name = ESP\_Client
\end{itemize}

\subsection{Optionnal: Change host name}
If \emph{hostname} return \emph{localhost.localdomain} you have to change host name before next step.
\begin{itemize}
\item Open a terminal
\item Connect as root
\item Open network file
	\newline
	\verb|>gedit /etc/sysconfig/network |
\item Replace in the file
	\newline
	\verb|HOSTNAME=localhost.localdomain |
	\newline
	with
	\newline
	\verb|HOSTNAME=ESP_Client (for example) |	
\item Save and close \emph{ /etc/sysconfig/network} file
\item Exit root access
	\newline
	\verb|>exit |
\item Reboot computer
\end{itemize}


\subsection{Update hosts file}
\begin{itemize}
\item Open a terminal
\item Connect as root
\item open hosts file
	\newline
	\verb|>gedit /etc/hosts |
\item append at the end of file
	\newline
	\verb|192.168.55.234  ESP_Client|
\item Save and close \emph{hosts} file
\item Exit root access
	\newline
	\verb|>exit |
\end{itemize}

\section{Remote configuration}
This step is mandatory each time a simulation has to be started with a new server computer (where EVC server will run) or a new client (where Test Environment will run).
\newline
To achieve this step you have to retrieve the server computer IP address (use command \emph{ifconfig} to get it) and host name (use command \emph{hostname} to get it).
\newline
To complete this step you have to know client computer IP address (use command \emph{ifconfig} to get it) and host name (use command \emph{hostname} to get it).
\newline
For example we assume:
\begin{itemize}
\item client ip address = 192.168.55.234
\item client host name = ESP\_Client
\item client account name = dbm
\item server ip address = 192.168.55.236
\item server host name = ESP\_Server
\item server account name = dbm
\end{itemize}
\subsection{Optional: Change host name}
\emph{Complete this step both on server computer and client computer}
If \emph{hostname} return \emph{localhost.localdomain} you have to change host name before next step.
\begin{itemize}
\item Open a terminal
\item Connect as root
\item Open network file
	\newline
	\verb|>gedit /etc/sysconfig/network |
\item Replace in the file
	\newline
	\verb|HOSTNAME=localhost.localdomain |
	\newline
	with
	\newline
	\verb|HOSTNAME=ESP_Client (or ESP_Server) |	
\item Save and close \emph{hosts} file
\item Exit root access
	\newline
	\verb|>exit |
\item Reboot computer
\end{itemize}

\subsection{Update hosts file}
\emph{Complete this step on server computer and client computer both}
\begin{itemize}
\item Open a terminal
\item Connect as root
\item open hosts file
	\newline
	\verb|>gedit /etc/hosts |
\item append (or edit if needed) at the end of file
	\newline
	\verb|192.168.55.236  ESP_Server|
	\newline
	\verb|192.168.55.234  ESP_Client|
\item Save and close \emph{hosts} file
\item Exit root access
	\newline
	\verb|>exit |
\end{itemize}

\label{ssh-key}
\subsection{Copy ssh key}

\subsubsection{Client side}
\begin{itemize}
\item Open a terminal
\item Generate ssh key
	\newline
	\verb|>ssh-keygen |
	\begin{itemize}
	\item For the question:
		\newline
		\verb|>Enter file in which to save the key (/home/dbm/.ssh/id_rsa):  |
		\newline
		press \emph{return}
	\item If command display the question: 
		\newline
		\verb|>: Overwrite (y/n)? |
		\newline
		press \emph{y}
	\item For all other questions press \emph{return}
	\end{itemize}
\item Copy ssh key on server
	\newline
	\verb|>ssh-copy-id dbm@192.168.55.236|		
	\begin{itemize}
	\item For the question:
		\newline
		\verb|>Are you sure you want to continue connecting (yes/no)?   |
		\newline
		enter \emph{yes}	
	\item enter password for server computer account to complete copy
	\end{itemize}
\item reboot computer
\end{itemize}

\subsubsection{Server side}
\begin{itemize}
\item Open a terminal
\item Generate ssh key
	\newline
	\verb|>ssh-keygen |
	\begin{itemize}
	\item For the question:
		\newline
		\verb|>Enter file in which to save the key (/home/dbm/.ssh/id_rsa):  |
		\newline
		press \emph{return}
	\item If command display the question : 
		\newline
		\verb|>: Overwrite (y/n)? |
		\newline
		press \emph{y}
	\item For all other questions press \emph{return}
	\end{itemize}
\item Copy ssh key on client computer
	\newline
	\verb|>ssh-copy-id dbm@192.168.55.234|		
	\begin{itemize}
	\item For the question:
		\newline
		\verb|>Are you sure you want to continue connecting (yes/no)?   |
		\newline
		enter \emph{yes}	
	\item enter password for client computer account to complete copy
	\end{itemize}
\item reboot computer
\end{itemize}

\subsection{Open communication port}
\emph{Complete this step on server computer only}
\label{trusted-network}
\subsubsection{Trusted network}
In order to allow communication between server computer and client computer you can disable firewall
On Centos : 
\begin{itemize}
\item Click on \emph{System}
\item Click on \emph{Administration}
\item Click on \emph{Firewall}
\item Enter server computer password (root password)
\item Click on \emph{Trusted Interfaces} (left panel)
\item Check \emph{eth0} and \emph{eth1} lines
\item Click on \emph{Apply} (top panel)
\item Close window
\end{itemize}

\label{untrusted-network}
\subsubsection{Untrusted network}
In order to allow communication between server computer and client computer (and only this communication) it is necessary to edit iptables.
For this step, values to change depend on current iptables configuration. In general, you have to use following command
\begin{itemize}
\item Open a terminal
\item Connect as root
\item Display current iptables configuration:
\newline
\verb|>iptables -nvL --line-numbers  |
\newline
For test computer we get following answer
\begin{lstlisting}[frame=single]

Chain INPUT (policy ACCEPT 0 packets, 0 bytes)
num pkts bytes target prot opt in out  source    destination         
1   6619 6544K ACCEPT all  --  *    *  0.0.0.0/0 0.0.0.0/0 ...
2      2   120 ACCEPT icmp --  *    *  0.0.0.0/0 0.0.0.0/0           
3     10   575 ACCEPT all  --  lo   *  0.0.0.0/0 0.0.0.0/0           
4     15   900 ACCEPT tcp  --  *    *  0.0.0.0/0 0.0.0.0/0 ...
5  14809 1514K REJECT all  --  *    *  0.0.0.0/0 0.0.0.0/0 ...
																														

Chain FORWARD (policy ACCEPT 0 packets, 0 bytes)
num pkts bytes target prot opt in out source    destination         
1      0     0 REJECT all  --  *    * 0.0.0.0/0 0.0.0.0/0 ...

Chain OUTPUT (policy ACCEPT 3929 packets, 1066K bytes)
num pkts bytes target prot opt in out source    destination
\end{lstlisting}
\item You have to insert authorization for client packets before REJECT all rules (5) in \emph{Chain INPUT} and \emph{Chain OUTPUT} if a rule exist for this section
\item to allow input from client, you have to insert at position 5 in iptables
 \newline
 \verb|>iptables -I INPUT 5 -s 192.168.55.234 -j ACCEPT  | 
\item to allow output to client, and as there is no rule REJECT all you can just add a rule to allow this communication
 \newline
 \verb|>iptables -A OUTPUT -d 192.168.55.234 -j ACCEPT  |  
\item Display the final iptables
\newline
\verb|>iptables -nvL --line-numbers  |
\newline
For test computer we get following answer
\begin{lstlisting}[frame=single]
Chain INPUT (policy ACCEPT 0 packets, 0 bytes)
num pkts bytes target prot opt in out source        dest         
1   6619 6544K ACCEPT  all --  *    * 0.0.0.0/0     0.0.0.0/0 
2      2   120 ACCEPT icmp --  *    * 0.0.0.0/0     0.0.0.0/0           
3     10   575 ACCEPT  all --  lo   * 0.0.0.0/0     0.0.0.0/0           
4     15   900 ACCEPT  tcp --  *    * 0.0.0.0/0     0.0.0.0/0
5      0     0 ACCEPT  all --  *    * 192.168.99.78 0.0.0.0/0           
6  15831 1628K REJECT  all --  *    * 0.0.0.0/0     0.0.0.0/0 

Chain FORWARD (policy ACCEPT 0 packets, 0 bytes)
num pkts bytes target prot opt in out source     dest         
1      0     0 REJECT  all --  *    * 0.0.0.0/0  0.0.0.0/0

Chain OUTPUT (policy ACCEPT 0 packets, 0 bytes)
num pkts bytes target prot opt in out source    dest        
1      0     0 ACCEPT  all --  *    * 0.0.0.0/0 192.168.99.78  
\end{lstlisting}
\item Save iptables
\newline
\verb|>service iptables save  |
\end{itemize}
To remove previous configuration, in order to avoid to keep useless open connection, you have to use following commands
\begin{itemize}
\item Remove in \emph{Chain INPUT}
 \newline
 \verb|>iptables -D INPUT -s 192.168.55.234 -j ACCEPT  | 
\item Remove in \emph{Chain OUTPUT}
 \newline
 \verb|>iptables -D OUTPUT -d 192.168.55.234 -j ACCEPT  | 
\end{itemize}

\chapter{User guide}
\section{Launch ESP Basic}
\subsection{In local configuration}
In order to launch ESP Basic in local configuration, you have to:
\begin{itemize}
\item Open a terminal
\item Go in installation folder
 \newline
 \verb|>cd /usr/local/ESP_Basic/test_runner2  | 
\item Launch script to run ESP\_Basic
 \newline
 \verb|>python RunTestRunner.py scenarios/ScenarioName.sce  | 
\end{itemize}
As result you can see a two GUIs displayed : 
\begin{itemize}
\item In a first time EVC\_Server
\item In a second time AutoTestRunner2 (Test Environment)
\end{itemize}

\subsection{In remote configuration}
In order to launch ESP Basic in remote configuration, proceed the following:
\begin{itemize}
\item Open a terminal on the client computer
\item Go in installation folder
 \newline
 \verb|>cd /usr/local/ESP_Basic/test_runner2  | 
\item Launch script to run ESP\_Basic
 \newline
 \verb|>python RunTestRunner.py -i 192.168.55.236 scenarios/ScenarioName.sce  | 
 \newline
	You have to specify in parameter server computer IP address.
\end{itemize}
As result you can see a two GUIs displayed: 
\begin{itemize}
\item In a first time EVC\_Server on the server computer
\item In a second time AutoTestRunner2 (Test Environment) on the client computer
\end{itemize}

\section{Close ESP Basic}
In order to close properly ESP Basic, you have to:
\begin{itemize}
\item Close AutoTestRunner2
\item Close EVC Server
\end{itemize}

\section{Available options}
Some options are available in the script \emph{RunTestRunner.py}
\begin{itemize}
\item -i: specify server computer ip address for TCP/IP communication (only in remote configuration). Default value is 127.0.0.1 and corresponds to the localhost. This will use your local IP stack.
\item -p: specify server computer port for TCP/IP communication (only in remote configuration). Default value is 12456.
\item -c: activate auto-close option. When scenario ended, Test Environment and EVC Server close automatically.
\item -d: generate debug files. When this option is not activated, ESP Basic generate debug files only for function which fail.
\item -s: specify path where is installed EVC Server (if EVC Server installation folder was moved). Default value is \emph{/usr/local/ESP\_Basic/evc\_server\_gui}
\item -x: activate Server X management. This option allows to run EVC Server on a server X (use in remote configuration only).
\item -m: disable automatic configuration checking. This allow to run EVC server when configuration has been done manually.
\end{itemize}
\section{GUI description}
\subsection{EVC Server}
	\begin{figure}[!h]
		\centering
		\includegraphics{DBM_03/EVC_Server_GUI.png}
		\caption{EVC Server}
		\label{fig:EVC Server}
	\end{figure}
	
\begin{itemize}
\item 1: Simulation information
\item 2: EVC TIU
\item 3: EVC server console
\item 4: Curves displaying
\item 5: Exit
\end{itemize}

\subsection{Test Environment}
	\begin{figure}[!h]
		\centering
		\includegraphics[width=\textwidth]{DBM_03/AutoTestRunner2.png}
		\caption{Test Environment}
		\label{fig:Test Environment}
	\end{figure}
	
\begin{itemize}
\item 1: Simplified DMI
\item 2: Scenario messages console
\item 3: Train TIU
\item 4: EVC TIU
\item 5: On-board status
\item 6: Running scenario
\item 7: Communication with server computer parameters
\end{itemize}

\section{Output files} 
\subsection{Server side}
There are 3 kinds of files generated during simulation:
\begin{itemize}
\item EVC log files in \emph{/usr/local/ESP\_Basic/evc\_server\_gui/data/log}
\item EVC curve images in \emph{/usr/local/ESP\_Basic/evc\_server\_gui/data/curve}
\item Potential error files (syntax debug\_*.stderr and debug\_*.stdout) in \emph{/usr/local/ESP\_Basic/evc\_server}
\end{itemize}
\subsection{Client side}
There is 1 kind of files generated during simulation
\begin{itemize}
\item Potential error files (syntax debug\_*.stderr and debug\_*.stdout) in \emph{/usr/local/ESP\_Basic/test\_runner2}
\end{itemize}
\chapter{Common problems and solutions}
\section{Client side displays : Fail to get x11-display}
Test Environment fails to get display address. To solve this issue:
\subsection{Solution 1}
Retry the previous command. Sometimes it may happen that the first time ESP\_Basic fail to get X11-display.
\subsection{Solution 2}
Activate or remove option -x in RunTestRunner command line.
\subsection{Solution 3}
On client computer and server computer, enter the command:
\newline
\verb|>xhost +  | 
\newline
If it helped to work, append this command in file \emph{/home/{username}/.bash\_profile} on client computer and server computer.
\section{Client side asks password}
Copy of the ssh key fail or ssh key is incorrect. To solve this issue:
\subsection{Solution 1}
Reboot both the client computer and server computer
\subsection{Solution 2}
Restart \ref{ssh-key}
\newline
\emph{Warning: Answer yes to question overwrite}
\section{EVC Server is launch but Test Environment not}
\subsection{File /etc/hosts is not correct}
Check IP address and hostname are correct in file \emph{etc/hosts}
\newline
Check computer host name is not present in line 127.0.0.1 and in line ::1. Remove it if you can find it in these lines.
\subsection{Iptables are not correct}
Your IP tables may be incorrect:
\newline
Try to disable firewall (3.2.4 A). If it works now your IP tables are wrong, restart 3.2.4 B
and check the position where you insert your rule (\emph{before rule with REJECT all}). Do not forget to save your iptables configuration.