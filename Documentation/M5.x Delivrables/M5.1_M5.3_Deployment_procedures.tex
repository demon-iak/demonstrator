\documentclass{template/openetcs_article}
% Use the option "nocc" if the document is not licensed under Creative Commons
%\documentclass[nocc]{template/openetcs_article}
\usepackage{lipsum,url}
\usepackage{supertabular}
\usepackage{multirow}
\usepackage{color, colortbl}
\definecolor{gray}{rgb}{0.8,0.8,0.8}
\usepackage[modulo]{lineno}
\graphicspath{{./template/}{.}{./images/}}

\begin{document}
\frontmatter
\project{openETCS}

%Please do not change anything above this line
%============================



% The document metadata is defined below

%assign a report number here
\reportnum{OETCS/WP5/M5.1}

%define your workpackage here
\wp{Work-Package 5: ``Demonstrator''}

%set a title here
\title{Installation guide for WP5 Demonstrator}

%set a subtitle here
%\subtitle{A template for short documents. Adapted from report template.}

%set the date of the report here
\date{June 2014}


%document approval
%define the name and affiliation of the people involved in the documents approbation here
\creatorname{Alexis Julin}
\creatoraffil{ERSA}

\techassessorname{Alexis Julin}
\techassessoraffil{ERSA}

\qualityassessorname{Ainhoa Gracia}
\qualityassessoraffil{SQS}

\approvalname{Klaus-R\"udiger Hase}
\approvalaffil{DB Netz}


%define a list of authors and their affiliation here

\author{Nicolas Van Landeghem, Didier Weckmann, Alexis Julin}

\affiliation{ERSA\\
5 Rue Maurice Blin \\
67500 Haguenau, France}



% define the coverart
\coverart[width=350pt]{openETCS_EUPL}

%define the type of report
\reporttype{Description of work}


%\begin{abstract}
%define an abstract here
%  \lipsum[12-13]
%\end{abstract}

%=============================
\maketitle

%Modification history
%if you do not need a modification history table for your document simply comment out the eight lines below
%=============================


\section*{Modification History}
\tablefirsthead{
\hline
\rowcolor{gray}
Version & Section & Modification / Description & Author \\\hline}
\begin{supertabular}{| m{1.2cm} | m{1.2cm} | m{6.6cm} | m{4cm} |}
 1.0.0&All Parts &New Document &Alexis Julin \\\hline
 1.0.1&New Part &Add license management &Alexis Julin \\\hline
 1.0.2&New Part &Add GE comments (issue nr.12) &Nicolas Van Landeghem \\\hline
\end{supertabular}


\tableofcontents
\newpage
%=============================

%Uncomment the next line if you need line numbers for tracebility when the document is in review
%\linenumbers
%=============================


% The actual document starts below this line
%=============================

%Start here
\section{Introduction}
This document is a guide for the installation and execution of the EVC simulator software libraries used by the Automatic Test Runner (ATR) aka Preliminary Test Environement.
\newline
\newline
EVC simulator software libraries is T5.2 deliverable (M5.1)
\newline
\newline
Automatic Test Runner (ATR) is T5.3 deliverable (M5.3)
\newpage
\section{Installation}
\subsection{Hardware and Software requirements}
OpenETCS step l deliverable has been installed and tested on a computer
running CentOS 6.5 operating system with the following mandatory conditions :
\begin{itemize}
	\item internet connection
	\item root access
\end{itemize}
Due to compatibility between Red Hat Enterprise versions and CentOS, installation works as well on Red Hat Enterprise Linux 6.
Step 1 deliverable is composed of M5.1 EVC kernel library which can be used through M5.3 Test environment executable. In a first step, the EVC kernel is not ready to host any external code, whereas the Test environment is provided for local use with some scenarios given as example.
\subsection{Installation}
\subsubsection{First steps}
\begin{itemize}
	\item Create a folder
	\item Copy OpenETCS archive in this folder
	\item Open a terminal
	\item Unpack OpenETCS archive
\newline	
\verb|>tar xvfj OpenETCS_WP5_X_X_X.tar.bz2 |
	\item Move to WP5\_Demonstrator\_Installer folder
\newline	
\verb|>cd WP5_Demonstrator_Installer |	
	\item Connect as root
\newline	
\newline NB: an archive is a compressed set of files provided by the project and/or the provider and that an end-user can uncompress or unpack onto its own computer.
\end{itemize}
\subsubsection{Mysql dependencies}
\begin{itemize}
	\item Install mysql-server package and its dependencies
	\newline
	\verb|>yum install mysql-server |
	\item Start mysql service
	\newline
	\verb|>service mysqld start |
\end{itemize}	
\subsubsection{SRS databases}
\begin{itemize}
	\item Install SRS baseline 230d database
	\newline
	\verb|>rpm -Uvh srs230d_db-1.0.3-220.i386.rpm |	
	\item Install SRS baseline 330 v11 database
	\newline
	\verb|>rpm -Uvh srs330_class1_v11_db-1.0.0-282.i386.rpm |	
	\item Install SRS baseline 330 v20 database
	\newline
	\verb|>rpm -Uvh srs330_class1_v20_db-1.0.0-282.i386.rpm |		
\end{itemize}	
\subsubsection{Qt driver}
\begin{itemize}
	\item Install driver mysql for Qt
	\newline
	\verb|>yum install qt-mysql |
\end{itemize}	

\subsubsection{OpenETCS}
\begin{itemize}
	\item Install OpenETCS step1 package
	\newline
	\verb|>yum localinstall openETCS-X.X.X-X.X.X.i386.rpm |
\end{itemize}	
\subsection{Uninstallation}
\begin{itemize}
	\item Uninstall OpenETCS step1 package
	\newline
	\verb|>rpm -e openETCS|
\end{itemize}
\subsection{Upgrade}
\begin{itemize}
	\item Upgrade OpenETCS step1 package
	\newline
	\verb|>yum localinstall openETCS-X.X.X-X.X.X.i386.rpm|
\end{itemize}	
\subsection{Downgrade}
\begin{itemize}
	\item Uninstall old OpenETCS step1 package
	\newline
	\verb|>rpm -e openETCS|
	\item Install new OpenETCS step1 package
	\newline
	\verb|>yum localinstall openETCS-X.X.X-X.X.X.i386.rpm|	
\end{itemize}		
\newpage
\section{Content of the package}
\subsection{OpenETCS archive}
A rpm (Redhat Packet Manager) is a OS-dedicated packet manager, used to create  automation when automatic deployment is required. It avoid end-users to have knowledge of which files have to be installed where on the Operating System. OpenETCS archive contains :
\begin{itemize}
	\item OpenETCS rpm
	\item SRS 230d database rpm
	\item SRS 330 v11 database rpm
	\item SRS 330 v20 database rpm
\end{itemize}
\subsection{Installation directory}
After installation procedure has been completed, following files and folders can be found in \textbf{/usr/local/openETCS} folder.
\subsubsection{Licensing folder}
Licensing folder contains all tools to manage license.
\subsubsection{lib folder}
Lib folder contains EVC simulator software libraries.
\subsubsection{test\_runner folder}
test\_runner folder contains :
\begin{itemize}
	\item scenarios folder which contains all (templates and user specific) Automatic Test Runner scenarios
	\item data folder which contains mainly simulation logs
	\item test\_runner, the Automatic Test Runner executable
\end{itemize}
\newpage
\section{Getting a license (First installation only)}
\subsection{Getting computer fingerprint}
\begin{itemize}
	\item open a \emph{non-root} terminal
	\item move to OpenETCS folder
	\newline
	\verb|>cd /usr/local/openETCS |	
	\item move to Licensing folder
	\newline
	\verb|>cd Licensing |
	\item launch script for getting fingerprint
	\newline
	\verb|> ./echoid >> echoid_output.txt|	
	\item send an email with file echoid\_output.txt to \emph{licenseactivation@ersa-france.com}
\end{itemize}
\subsection{Installing license}
You will receive an email with a licensing file
\begin{itemize}
	\item open licensing file with a text editor
	\item open a root terminal
	\item move to OpenETCS folder
	\newline
	\verb|>cd /usr/local/openETCS |	
	\item move to Licensing folder
	\newline
	\verb|>cd Licensing |
	\item launch license installer script
	\newline
	\verb|>./licenseinstaller -i -l 'xxxx' /opt/ERSA/license/license.rc |	where xxxx must be replaced by the key in licensing file (you can find it from first character to \# character \emph{include})
\end{itemize}
\newpage
\section{Running the Automatic Test Runner}
To execute Automatic Test Runner : 
\begin{itemize}
	\item Open a terminal
	\item move to OpenETCS folder
	\newline
	\verb|>cd /usr/local/openETCS |		
	\item launch test\_runner
	\newline
	\verb|>./test_runner oETCS_scenarios/scenario_name.sce |	
\end{itemize}


%===================================================
%Do NOT change anything below this line

\end{document}